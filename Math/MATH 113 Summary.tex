\documentclass{article}
\usepackage[utf8]{inputenc}
\usepackage{amsmath,amsthm,amssymb,tikz-cd}
\usepackage{enumitem}
\usepackage{fullpage}
\usepackage{setspace}
\usepackage{mathtools}
\usepackage[bookmarks]{hyperref}
\newcommand{\R}{\mathbb{R}}
\newcommand{\Z}{\mathbb{Z}}
\newcommand{\C}{\mathbb{C}}
\newcommand{\F}{\mathbb{F}}
\newcommand{\ga}{\langle a \rangle}
\renewcommand{\phi}{\varphi}

\onehalfspacing
\setlength{\parskip}{0em}

\begin{document}

\begin{center}
    {\Large MATH 113 Summary of Facts} \vspace{0.5em}
    \\ Cheng, Feng
    \vspace{-0.5em}
\end{center}

\section{Groups and Subgroups}
\begin{itemize}
    \item Left and right cancellation laws in groups
    \item Unique solution to all linear equations, and thus every element appears exactly once in every row and column of a group table
    \item Unique identity and inverse
    \item $(ab)^{-1}= b^{-1}a^{-1}$
    \item There are one group of order 1/2/3, up to isomorphism. There are two groups of order 4, up to isomorphism, $\Z_4$ and Klein 4-group $V$. Groups of order less than 6 are all abelian. The smallest nonabelian group is of order 6, and $S_3$ is a common counterexample to many claims.
    \item The only idempotent element of a group is the identity.
    \item $\Z_4$ has one nontrivial proper subgroup $\{0,2\}$, while $V$ has three: $\{e,a\}, \{e,b\},\{e,c\}$.
    \item To check if $H \subseteq G$ is a subgroup, we check $H$ is closed, $H$ has an identity, and the inverse exists for every element of $H$. Associativity is inherited from the whole group.
    \item $H = \{a^n \mid n \in \Z\}$ is the smallest subgroup of $G$ containing $a \in G$, called the cyclic subgroup of $G$ generated by $a$. $H$ can be denoted by $\langle a \rangle$. If $G$ is the whole group, then $a$ generates $G$ and $G$ is a cyclic group.
    \item Every cyclic group is abelian.
    \item A subgroup of a cyclic group is cyclic by the division algorithm. This implies that the subgroups of $\Z$ are the groups $n\Z$ (they are all the cyclic subgroups of $\Z$ generated by elements of $\Z$).
    \item The positive generator of $H = \{ar+bs \mid a,b \in \Z\}$ under addition is $\gcd(r,s)$.
    \item If $G = \langle a \rangle$ is infinite, then it is isomorphic to $\Z$ under addition; if it is finite, then it is isomorphic to $\Z_n$ under $+_n$. In the proof we show that for finite $\langle a \rangle$, the smallest number $m$ such that $a^m = e$ is its order. For infinite $\langle a \rangle$, m does not exist.
    \item $G = \langle a \rangle$ of order $n$, then $a^s$ generates a cyclic subgroup of order $\frac{n}{\gcd(n,s)}$. Also $\langle a^s \rangle = \langle a^t \rangle$ iff $\gcd(s,n)=\gcd(t,n)$.
    \item It follows that $a$ generates $G$ of order $n$, then all other generators of $G$ are $a^r$, where $r$ and $n$ are coprime.
    \item Arbitrary intersection of subgroups $H_i$ of a group $G$ is a subgroup of $G$.
    \item The smallest subgroup of G containing $\{a_i \mid i \in I\}$ is the subgroup generated by this generating set. It has the elements of $G$ that are the product of integral powers of the $a_i$'s, and one $a_i$ may appear several times in the product because $G$ is not necessarily abelian.
\end{itemize}

\section{Permutation, Cosets, and Direct Products}
\begin{itemize}
    \item A permutation of a set $A$ is a bijective function $\phi: A \to A$.
    \item The set of permutation $S_A$ is a group under permutation multiplication (composition).
    \item Permutation $S_n$ on $\{1,2,\dots,n\}$ is isomorphic to all permutations of order $n$, and can thus be taken as the prototype for permutations of order $n$.
    \item $S_3 \simeq D_3$ under renaming
    \item For one-to-one function $\phi: G \to G'$ satisfying the homomorphism property, the image $\phi(G)$ is a subgroup of $G'$ and $\phi(G) \simeq G$. We use this to show Cayley's theorem that every group is isomorphic to a group of permutations.
    
    \item Lagrange's Theorem states that the order of a subgroup of a finite group is the divisor of the order of the whole group. This is because the cardinality of the cosets and the order of the subgroup are the same.
    \item A prime-order group must be cyclic. If one considers the cyclic subgroup generated by $a \not= e$, then $|\ga| \geq 2$ and thus must be $p$.
    \item The entry-wise direct product $\prod_{i} G_i$ of groups $G_i$ forms a new group.
    
\end{itemize}

\section{Homomorphisms and Factor Groups}
\begin{itemize}
    \item 
\end{itemize}

\end{document}