\documentclass{article}
\usepackage[utf8]{inputenc}
\usepackage{amsmath,amsthm,amssymb,tikz-cd}
\usepackage{enumitem}
\usepackage{fullpage}
\usepackage{setspace}
\usepackage{mathtools}
\usepackage[bookmarks]{hyperref}
\newcommand{\R}{\mathbb{R}}
\newcommand{\Z}{\mathbb{Z}}
\newcommand{\C}{\mathbb{C}}
\newcommand{\F}{\mathbb{F}}
\newcommand{\aut}{\operatorname{Aut}}
\newcommand{\inn}{\operatorname{Inn}}
\newcommand{\ga}{\langle a \rangle}
\renewcommand{\phi}{\varphi}

\onehalfspacing
\setlength{\parskip}{0em}

\begin{document}

\begin{center}
    {\Large MATH 113 Summary of Facts} \vspace{0.5em}
    \\ Cheng, Feng
    \vspace{-0.5em}
\end{center}

\section{Groups and Subgroups}
\begin{itemize}
    \item Left and right cancellation laws in groups
    \item Unique solution to all linear equations, and thus every element appears exactly once in every row and column of a group table
    \item Unique identity and inverse
    \item $(ab)^{-1}= b^{-1}a^{-1}$
    \item There are one group of order 1/2/3, up to isomorphism. There are two groups of order 4, up to isomorphism, $\Z_4$ and Klein 4-group $V$. Groups of order less than 6 are all abelian. The smallest nonabelian group is of order 6, and $S_3$ is a common counterexample to many claims.
    \item The only idempotent element of a group is the identity.
    \item $\Z_4$ has one nontrivial proper subgroup $\{0,2\}$, while $V$ has three: $\{e,a\}, \{e,b\},\{e,c\}$.
    \item To check if $H \subseteq G$ is a subgroup, we check $H$ is closed, $H$ has an identity, and the inverse exists for every element of $H$. Associativity is inherited from the whole group.
    \item $H = \{a^n \mid n \in \Z\}$ is the smallest subgroup of $G$ containing $a \in G$, called the cyclic subgroup of $G$ generated by $a$. $H$ can be denoted by $\langle a \rangle$. If $G$ is the whole group, then $a$ generates $G$ and $G$ is a cyclic group.
    \item Every cyclic group is abelian.
    \item A subgroup of a cyclic group is cyclic by the division algorithm. This implies that the subgroups of $\Z$ are the groups $n\Z$ (they are all the cyclic subgroups of $\Z$ generated by elements of $\Z$).
    \item The positive generator of $H = \{ar+bs \mid a,b \in \Z\}$ under addition is $\gcd(r,s)$.
    \item If $G = \langle a \rangle$ is infinite, then it is isomorphic to $\Z$ under addition; if it is finite, then it is isomorphic to $\Z_n$ under $+_n$. In the proof we show that for finite $\langle a \rangle$, the smallest number $m$ such that $a^m = e$ is its order. For infinite $\langle a \rangle$, m does not exist.
    \item $G = \langle a \rangle$ of order $n$, then $a^s$ generates a cyclic subgroup of order $\frac{n}{\gcd(n,s)}$. Also $\langle a^s \rangle = \langle a^t \rangle$ iff $\gcd(s,n)=\gcd(t,n)$.
    \item It follows that $a$ generates $G$ of order $n$, then all other generators of $G$ are $a^r$, where $r$ and $n$ are coprime.
    \item Arbitrary intersection of subgroups $H_i$ of a group $G$ is a subgroup of $G$.
    \item The smallest subgroup of G containing $\{a_i \mid i \in I\}$ is the subgroup generated by this generating set. It has the elements of $G$ that are the product of integral powers of the $a_i$'s, and one $a_i$ may appear several times in the product because $G$ is not necessarily abelian.
\end{itemize}

\section{Permutation, Cosets, and Direct Products}
\begin{itemize}
    \item A permutation of a set $A$ is a bijective function $\phi: A \to A$.
    \item The set of permutation $S_A$ is a group under permutation multiplication (composition).
    \item Permutation $S_n$ on $\{1,2,\dots,n\}$ is isomorphic to all permutations of order $n$, and can thus be taken as the prototype for permutations of order $n$.
    \item $S_3 \simeq D_3$ under renaming
    \item For one-to-one function $\phi: G \to G'$ satisfying the homomorphism property, the image $\phi(G)$ is a subgroup of $G'$ and $\phi(G) \simeq G$. We use this to show Cayley's theorem that every group is isomorphic to a group of permutations.
    
    \item Lagrange's Theorem states that the order of a subgroup of a finite group is the divisor of the order of the whole group. This is because the cardinality of the cosets and the order of the subgroup are the same.
    \item A prime-order group must be cyclic. If one considers the cyclic subgroup generated by $a \not= e$, then $|\ga| \geq 2$ and thus must be $p$.
    \item The number of left cosets of $H$ in $G$ is the \textit{index} $(G:H)$ of $H$ in $G$. The product rule $(G:K) = (G:H)(H:K)$ holds for a chain of subgroup, regardless of whether the groups involved are finite or not.
    \item The entry-wise direct product $\prod_{i} G_i$ of groups $G_i$ forms a new group. Rearrangement of factors yields the same group of direct product, up to isomorphism.
    \item $\Z_m \times \Z_n$ is cyclic and thus isomorphic to $\Z_{mn}$ if and only if $\gcd(m,n) = 1$. Obviously this can be extended to direct products of higher dimensions.
    \item For $(a_1, \dots, a_n) \in \prod_{i=1}^n G_i$, the order of this element is the lcm of the orders of each individual $a_i$'s.
    \item The fundamental theorem of finitely generated abelian groups states that every finitely generated abelian groups $G$ is isomorphic to a direct product of cyclic groups in the form $$\Z_{(p_1)^{r_1}} \times \Z_{(p_2)^{r_2}} \times \cdots \times \Z_{(p_n)^{r_n}} \times \Z \times \Z \times \cdots \times \Z,$$ where the $\Z$'s at the end have a fixed number called the \textit{Betti number} of $G$, and the $p_j$'s at the front do not need to be distinct.
    \item A group $G$ is \textit{decomposable} if it is isomorphic to a direct product of two nontrivial subgroups. The finite indecomposable abelian groups are exactly the ones with order a power of a prime.
    \item If $m \mid |G|$, where $G$ is a \textbf{finite abelian group}, then $G$ has a subgroup of order $m$. Consider $$\langle (p_1)^{r_1 - s_1} \rangle \times \langle (p_2)^{r_2 - s_2} \rangle \times \cdots \times \langle (p_n)^{r_n - s_n} \rangle.$$
    \item If $m$ is a square-free integer, then abelian group of order $m$ is cyclic.
\end{itemize}

\section{Homomorphisms and Factor Groups}
\begin{itemize}
    \item A surjective group homomorphism $\phi$ from an abelian group $G$ to another group $G'$ tells us $G'$ is abelian. (This is the preservation of algebraic structure.)
    \item It suffices to show from the homomorphism property that $\phi(e) = e_G$, $\phi(a^{-1}) = (\phi(a))^{-1}$.
    \item The image of a subgroup of $G$ is a subgroup of $G'$. The preimage of a subgroup of $G'$ is a subgroup of $G$. It follows directly that the kernel and image of a group homomorphism is the respective subgroups of $G$ and $G'$. Furthermore, $\ker \phi$ is a normal subgroup of $G$.
    \item $\phi$ is one-to-one if and only if its kernel only contains $e$. Thus, to prove isomorphism, one checks the homomorphism property, the kernel, and onto.
    \item There are several equivalent conditions to check that a subgroup is normal. Also, every subgroup of an abelian group is normal.
    \item The coset multiplication operation on $H \leq G$ is well-defined if and only if $H \trianglelefteq G$.
    \item Given $\phi: G \to H$ and the natural surjective homomorphism $\pi: G \to G/N$ with kernel $N$, $\tilde{\phi}: G/N \to H$ defined by $$\tilde{\phi}(gN) = \phi(g)$$ is well-defined if and only if $N \leq \ker \phi$.
    \item The full version of the fundamental theorem of group homomorphism states that, given a group homomorphism $\phi: G \to H$, let $N$ be a normal subgroup in $G$ and $\pi: G \to G/N$ be the natural surjective homomorphism, if $N \leq \ker \phi$, then there exists a unique homomorphism $\tilde{\phi}: G/N \to H$ such that $\phi = \tilde{\phi} \circ \pi$.
    \begin{itemize}
        \item In particular, if we let $\ker \pi = N = \ker \phi $, then $\ker \tilde{\phi}$ contains only the identity an it thus injective. Therefore, $\tilde{\phi}: G/N \to \phi(G)$ is an isomorphism.
    \end{itemize}
    \item The automorphism group $\aut(G) \coloneqq \{\phi: G \to G \mid \phi \text{ is an isomorphism}\}$. In particular $i_g: G \to G$ given by $i_g(x) = g x g^{-1}$ is an automorphism of $G$, called the inner automorphism of $G$. These $i_g$'s form $\inn(G) \coloneqq \{\phi \in \aut(G) \mid \phi = i_g \text{ for some } g \in G\}$, which turns out to be a normal subgroup of $\aut(G)$.
    \item $\operatorname{Out}(G) \coloneqq \aut(G)/\inn(G)$.
    \item $\phi: G \to \inn(G)$ is a surjective group homomorphism with kernel $Z(G) \coloneqq \{g \in G \mid g x g^{-1} = x \text{ for all } x \in G\}$, called the \textit{center} of $G$. Hence $G/Z(G) \simeq \inn(G)$.
\end{itemize}

\end{document}