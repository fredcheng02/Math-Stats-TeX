\documentclass{article}
\usepackage[utf8]{inputenc}
\usepackage{amsmath,amsthm,amssymb}
\usepackage{enumitem}
\usepackage{fullpage}
\usepackage{setspace}

\onehalfspacing
\pagestyle{empty}
\setlength{\parskip}{0.5em}

\begin{document}

\begin{center}
    \Large Equivalency between Darboux and Riemann Integrals
\end{center}

\begin{proof}
    Recall the Cauchy condition states that a function $f$ is Darboux integrable iff 
    \begin{equation}
        \forall \epsilon>0, \exists \text{ a partition } P_\epsilon \text{ such that } U(f, P_\epsilon) - L(f, P_\epsilon) < \epsilon.
    \end{equation}
    We may first show that the Riemann integral definition is equivalent to saying that
    \begin{equation}
        \forall \epsilon > 0, \exists \delta > 0 \text{ such that for any partition } P \text{ with } \|P\| < \delta, U(f, P) - L(f, P) < \epsilon,
    \end{equation}
    which we will then show is equivalent to the Cauchy condition above. 
    
    We write the Riemann sum based on $P$ with $n$ subintervals as $S = \sum_{i = 1}^{n} f(t_i) \Delta x_i$, where $t_i$ can take any value within $[ x_{i-1}, x_{i}]$. Riemann integral exists when this $S$ converges to a limit $T$ as $\|P\| \to 0$.
    
    \emph{Step I.} Riemann integral $\Longleftrightarrow$ (2).
    Suppose $f$ is Riemann integrable, then $$\forall \epsilon > 0, \exists \delta > 0 \text{ such that for any partition } P \text{ with } \|P\| < \delta, |S - T| < \frac{\epsilon}{2}.$$
    As a result, $T - \frac{\epsilon}{2} < S < T + \frac{\epsilon}{2}$, and taking the infimum and supremum of $S$ over $t_i$'s gives us $T - \frac{\epsilon}{2} \leq L(f,P)$ and $T + \frac{\epsilon}{2} \geq U(f,P)$. Therefore, $U(f,P) - L(f,P) \leq \epsilon$ for any fixed $P_{\epsilon,\delta}$.
    
    Conversely, $L(f,P) < S < U(f,P)$, and we only have to show that a constant $T$ exists between $U(f,P)$ and $L(f,P)$ for arbitrary $P$ with $\|P\| < \delta$, so that $|S - T| < U(f,P) - L(f,P) < \epsilon$. This is always true because $U(f,P) \geq L(f,Q)$ for any possible partitions $P$ and $Q$.
    
    \emph{Step II.} (2) $\Longleftrightarrow$ (1). The $\Longleftarrow$ direction is trivial because the partition we get from (2) is the partition that satisfies (1) as well.
    The other direction is more involved. The basic idea is to construct any partition $P$ with $\|P\| < \delta$ from the existing $P_\epsilon$ from the Cauchy condition.
    
    Let $P_\epsilon = \{y_0, y_1,...,y_N\}$ be the partition in (1), and let $K = \sup \{|f(x)|: x \in [a,b]\}$ so that $M_i - m_i \leq 2K$ in each of the $N$ subintervals. Also, let $\delta = \frac{\epsilon}{4NK}$. Now for any partition $P = \{x_0, x_1, ..., x_n\}$ with $\|P\| < \delta$, we divide the sum $$U(f,P) - L(f,P) = \sum_{i=1}^n (M_i - m_i) (x_i - x_{i-1})$$ into two categories based on the intervals. First, we have terms with intervals $[x_{i-1}, x_i] \subseteq [y_{j-1}, y_j]$ for some $j$. The sum of these terms $$S_1 \leq U(f, P_\epsilon) - L(f, P_\epsilon) < \frac{\epsilon}{2}.$$ For the intervals that are left out, they must necessarily contain points of $P_\epsilon$ in it. Therefore, the number of intervals is at most $N$, and it follows that the sum of the corresponding terms $$S_2 < N \cdot 2K \cdot \delta = \frac{\epsilon}{2}.$$ And thus we have $U(f,P) - L(f,P) = S_1 + S_2 < \epsilon$ for any partition $P$ with $\|P\| < \delta$.
\end{proof}

\end{document}