\documentclass[11pt]{article}
\usepackage[T1]{fontenc}
\usepackage{amsmath,amssymb,amsthm}
\usepackage{mathtools}
\usepackage{enumitem}
\usepackage[margin=1in]{geometry}
\usepackage{setspace,microtype,soul,framed}
\usepackage[pdfusetitle,bookmarksnumbered]{hyperref}
\usepackage{mathpazo}
\usepackage{eucal}
\usepackage[scaled=1.008]{inconsolata}
\usepackage[scaled=0.842]{berasans}

\newcommand{\lk}[2]{\hyperlink{subsection.#1.#2}{\S#1.#2}}
\newcommand{\rmk}{\noindent\textit{Remark. }}
\newcommand{\df}[1]{\textit{\textsf{#1}}} % definition
\newcommand{\la}{\langle}
\newcommand{\ra}{\rangle}
% \usepackage{bbm} use \mathbbm for blackboard bold
\newcommand{\N}{\mathbf{N}}
\newcommand{\Z}{\mathbf{Z}}
\newcommand{\Q}{\mathbf{Q}}
\newcommand{\R}{\mathbf{R}}
\newcommand{\C}{\mathbf{C}}
\newcommand{\B}{\mathcal{B}} % Borel
\newcommand{\F}{\mathcal{F}} % Sigma Field
\newcommand{\A}{\mathcal{A}}
\newcommand{\G}{\mathcal{G}}
\newcommand{\where}{\,|\,}
\newcommand{\clos}[1]{\overline{#1}}
\newcommand{\inp}[2]{\langle #1, #2 \rangle}
\newcommand{\nm}[1]{\lVert #1 \rVert}
\newcommand{\abs}[1]{\lvert #1 \rvert}
\newcommand{\upX}{\bar X}
\newcommand{\lowX}{\underline{\mathrm X}}
\newcommand{\Var}{\operatorname{Var}}
\newcommand{\Cov}{\operatorname{Cov}}
\newcommand{\Corr}{\operatorname{Corr}}
\newcommand{\convd}{\xrightarrow[]{\text D}}
\newcommand{\convp}{\xrightarrow[]{\text P}}
\newcommand{\convas}{\xrightarrow[]{\text{a.s.}}}

\usepackage{xcolor}
\newcommand{\tpurp}[1]{\textcolor{purple}{#1}}

\usepackage{tikzit}
\input{FtoBsty.tikzstyles}

\makeatletter
\let\@subtitle\@empty % default value
\protected\def\subtitle#1{\gdef\@subtitle{#1}}
\def\@maketitle{%
  \newpage
  \begin{center}%
  \let \footnote \thanks
    {\Large \@title \par}%
    {\@subtitle \par}%
    \vskip 0.5em%
    {\lineskip .5em%
      \begin{tabular}[t]{c}%
        \@author
      \end{tabular}\par}%
    {\@date}%
    \vskip -0.5em%
  \end{center}%
  \par}
\makeatother

\title{MATH 106 Companion}
\author{Cheng, Feng}
\subtitle{to Walsh's \\ \textit{Knowing the Odds \\ An Introduction to Probability}}
\date{}

\onehalfspacing
\setlength{\parskip}{0em}
\setlist{listparindent=\parindent,parsep=0pt}

\begin{document}
\maketitle

\section{Probability Space}
\begin{itemize}
    \item For a set $X$ and a $\sigma$-algebra $\A$ containing some subsets of $X$, a \df{measure} on $(X,\A)$ is a function $\mu: \A \to [0, \tpurp{\infty]}$ such that 
    \begin{enumerate}[label=(\roman*)]
        \item $\mu(\emptyset)=0$;
        \item for a list of disjoint sets $A_i$ from $\A$, $\mu(\cup_{i=1}^\infty A_i) = \sum_{i=1}^\infty \mu(A_i)$.
    \end{enumerate}
    If at least one set $C$ from $\A$ has finite measure, then we can show (i) by (ii): \[\mu(C) = \mu(C \cup \emptyset) = \mu(C) + \mu(\emptyset) \implies \mu(\emptyset) = 0.\]
    
    A \df{probability measure} on $(\Omega, \F)$ is a function $P: \F \to [0,\tpurp{1]}$ such that 
    \begin{enumerate}[label=(\roman*)]
        \item $P(\Omega) = 1$;
        \item for a list of disjoint events $A_i$ from $\F$, $P(\cup_{i=1}^\infty A_i) = \sum_{i=1}^\infty P(A_i)$.
    \end{enumerate}
    In addition to being a measure on $(\Omega, \F)$, a probability measure specifies the measure of $\Omega$ as 1. Because $P$ is finite, we can prove $P(\emptyset) = 0$ by (ii).
    
    Without specifying, we will be working in the probability space $(\Omega, \F, P)$ from now on.
\end{itemize}

\section{Random Variables}
\begin{itemize}
    \item A \df{random variable} is a function $X: \Omega \to \R$ such that for all $x \in \R$, $\{\omega : X(\omega) \leq x\} \in \F$. For intuitive understanding we write $\{\omega : X(\omega) \leq x\}$ mostly as $\{X \leq x\}$, and see $X$ as an actual random ``variable''.
    
    This is equivalent to saying that $X$ is a $\F$-measurable function. We know that the sets $(-\infty, x]$ for all $x \in \R$ generates $\B$. It suffices to check that $X^{-1}(-\infty, x]$ is $\F$-measurable by the fact that taking the preimage preserves set operations. This is true by our original definition.
    
    Proposition 2.8 in the book essentially says the same thing, that $X$ is a $\F$-measurable function that carries subsets of $\Omega$ from $\F$ to the Borel subsets of $\R$. Intuitively, this says that \emph{a random variable $X$ on a Borel set $A$ must be an event} (without specifying $\F$).
    \begin{itemize}
        \item Exercise 2.1(b). Prove that the sum of two random variables from the same probability space is a random variable.
        \begin{proof}
        First we show that \emph{for any arbitrary $x$}, \begin{equation}
            \{\omega : X(\omega) + Y(\omega) > x\} = \cup_{\tpurp{r \in \Q}}\{\omega : X(\omega) > r, Y(\omega) > x - r\}.
        \end{equation}
        Clearly the RHS of the expression we proved is a countable union over $r$ of $\{\omega : X(\omega) > r, Y(\omega) > x - r\}$, which is the intersection of $\{\omega : X(\omega) > r\}$ and $\{\omega : Y(\omega) > x - r\}$. These two are the complements of $\{\omega : X(\omega) \leq r\}$ and $\{\omega : Y(\omega) \leq x - r\}$ respectively. Because $X, Y$ are random variables, the two sets are in $\F$, and by the axioms of a $\sigma$-field we see that $\{\omega : X(\omega) + Y(\omega) \leq x\} = \{\omega : X(\omega) + Y(\omega) > x\}^c \in \F$ for all $x \in \R$. This proves that $X+Y$ is a random variable.
        \end{proof}
        Pay attention that the role of (1) in the proof. Taking the complement of $\{\omega : X(\omega) + Y(\omega) \leq x\}$ makes sense here because it allows to get a general expression of the set as a countable union over a parameter.
        \item Exercise 2.1(d).
    \end{itemize}
    \item The \df{distribution function of a random variable $X$} is the function $F_X: \R \to [0,1]$ given by \[F_X(x) = P(\omega : X(\omega) \leq x) = P(X^{-1}(-\infty, x]).\] It is important to note that the distribution function at $x$ is the probability measure on the event $\{\omega : X(\omega) \leq x\}$ [or $\{X \leq x\}$, using the shorthand notation]. We need to specify that this set $\{\omega : X(\omega) \leq x\}$ is from $\F$ so that the probability measure on it makes sense. The definition of $F_X$ justifies the vague definition of a random variable.
    \item Clearly if $A \in \F$, then $I_A$ is a random variable. We now show that if $I_A$ is a random variable, then $A \in \F$.
    
    Consider $\{\omega : I_A(\omega) \leq x\} = 
    \begin{cases}
        \emptyset & \text{if } x < 0, \\
        A^c & \text{if } 0 \leq x < 1, \\
        \Omega & \text{if } x \geq 1.
    \end{cases}$ By the definition of a random variable $A^c \in \F$, meaning that $A \in \F$.
    \item The \df{distribution} of a random variable is the function $\mu: \B \to [0,1]$ given by \[\mu(A) = P(X \in A) = P(X^{-1}(A)).\]
    The definition makes sense because we know that a random variable on a Borel set always has probability on it.

    What is important about this definition is that through $X$, $\mu$ becomes a new probability measure on $(\R, \B)$: $\mu(A)$ evaluate the probability that the random variable $X$ falls in $A$.

    We now verify that $\mu$ is a probability measure. First, $\mu$ is nonnegative because $P$ is nonnegative. Second, countable additivity follows straight from the preservation of set operations under preimage: for disjoint Borel sets $A_1,A_2,\dots,$ \[\mu(\cup_i A_i) = P(X^{-1}(\cup_i A_i)) = P(\cup_i X^{-1}(A_i)).\] Because preimages of disjoint sets are disjoint, by $P$ being a measure we have \[\textstyle P(\cup_i X^{-1}(A_i)) = \sum_i P(X^{-1}(A_i)) = \sum_i \mu(A_i).\] Furthermore, $\mu(\R) = P(X^{-1}(\R)) = P(\Omega) = 1$. This completes the proof.
    
    The following figure illustrates the connection between the probability space $(\Omega,\F,P)$ and the induced probability space $(\R,\B,\mu)$.`
    \ctikzfig{FtoBfig}
    Note that this is a special setting of the \df{pushforward measure}. A new measure is obtained by pushing forward the original measure on one measure space to another space via the preimage of a measurable function.
    
\end{itemize}
\end{document}