\documentclass[11pt]{article}
\usepackage[T1]{fontenc}
\usepackage[margin=1in]{geometry}
\usepackage{amsmath,amssymb,amsthm}
\usepackage{mathtools}
\usepackage{enumitem}
\usepackage{setspace,microtype}
\usepackage{framed,xcolor}
\usepackage[pdfusetitle,bookmarksnumbered,colorlinks]{hyperref} %load this at the end

\newcommand{\la}{\langle}
\newcommand{\ra}{\rangle}
\newcommand{\R}{\mathbf{R}}
\newcommand{\where}{\,|\,}
\newcommand{\blank}{\,\cdot\,}
\newcommand{\inp}[2]{\langle #1, #2 \rangle}
\newcommand{\nm}[1]{\lVert #1 \rVert}
\newcommand{\abs}[1]{\lvert #1 \rvert}
\newcommand{\ol}[1]{\overline{#1}}

\renewcommand{\Pr}{\operatorname{P}}
\newcommand{\E}{\mathbf{E}}
\newcommand{\I}[1]{\mathbf{1}_{\{#1\}}}
\newcommand{\SE}{\operatorname{SE}}
\newcommand{\Var}{\operatorname{Var}}
\newcommand{\Cov}{\operatorname{Cov}}
\newcommand{\Corr}{\operatorname{Corr}}
\newcommand{\argmax}{\operatorname*{arg\,max}}
\newcommand{\argmin}{\operatorname*{arg\,min}}
\newcommand{\convd}{\xrightarrow[]{\text D}}
\newcommand{\convp}{\xrightarrow[]{\text P}}
\newcommand{\convas}{\xrightarrow[]{\text{a.s.}}}
\newcommand{\eqD}{\stackrel{\text D}{=}}

\newcommand{\df}[1]{\textit{\textsf{#1}}} % definition
% \newcommand{\codevar}[1]{\ensuremath{\mathtt{\textcolor{brown}{#1}}}} % variable in code

\usepackage{thmtools}
\declaretheoremstyle[headfont=\scshape]{plain}
\theoremstyle{plain}
\newtheorem{thm}{Theorem}
\newtheorem*{thm*}{Theorem}
\newtheorem*{lem*}{Lemma}
\declaretheoremstyle[headfont=\bfseries]{definition}
\theoremstyle{definition}
\newtheorem{defn}[thm]{Definition}
\declaretheoremstyle[headfont=\itshape]{remark}
\theoremstyle{remark}
\newtheorem{rmk}[thm]{Remark}
\theoremstyle{definition}
\newtheorem{sol}[thm]{Solution}
\newtheorem{mansolinner}{Solution}
\newenvironment{mansol}[1]{%
  \renewcommand\themansolinner{#1}%
  \mansolinner
}{\endmansolinner}


\usepackage{mathpazo}
\usepackage[scaled=1.008]{inconsolata}
\usepackage[scaled=0.842]{berasans}

\onehalfspacing
\setlength{\parskip}{0em}
\setlist{listparindent=\parindent,parsep=0pt}

\makeatletter
\let\@subtitle\@empty % default value
\protected\def\subtitle#1{\gdef\@subtitle{\\ #1}}
\renewcommand{\maketitle}{
    \begin{center}
        {\Large \@title}
        \@subtitle
        \vspace{0.5em}
        \\ \@author \\ \@date
        \vspace{-0.5em}
    \end{center}
}
\makeatother

\title{HDP Exercises}
\author{Cheng, Feng}
\date{}

\begin{document}
\maketitle

\begin{mansol}{1.2.2}
    We already know that for $X \geq 0$, \[\E X = \int_{0}^\infty \Pr(X > t) \, dt.\] In general for a real number $x$, \[
    x = \I{x \geq 0} \cdot \int_0^\infty \I{t < x} \,dt - \I{x < 0} \cdot \int_{-\infty}^0 \I{x < t} \,dt.
\]
When $x<0$, $\I{t<x} = 0$ for all $t\geq 0$; when $x \geq 0$, $\I{x<t} = 0$ for all $t \leq 0$. Hence \[x = \int_0^\infty \I{t < x} \,dt - \int_{-\infty}^0 \I{x < t} \,dt,
\]
and by interchanging $\int$ and $\E$ (via Fubini-Tonelli's theorem), we have \[ \E X = \int_{0}^\infty \Pr(X > t) \, dt - \int_{-\infty}^0 \Pr(x < t) \,dt. \qedhere\]
\end{mansol}

\begin{mansol}{1.2.3}
By the change of variables formula, since $g(t) = t^p$ is a strictly increasing function that maps $[0,\infty)$ onto $[0,\infty)$,
\begin{align*}
\text{RHS} & = \int_0^\infty \Pr(\abs X  > t) (t^p)'\,dt \\
& = \int_0^\infty \Pr(\abs X > \sqrt[n]{u})\,du \\ & = \int_0^\infty \Pr(\abs X ^p > u) \,du = \text{LHS}.
\end{align*}
\end{mansol}

\begin{mansol}{1.3.3}
Jensen's inequality tells us $(\E X)^2 \leq \E X^2$ [true regardless of whether $X$ has finite second moment or not]. Now consider \[
\Bigl(\E \Big\lvert\frac{1}{N} \sum_{i=1}^N X_i - \mu\Big\rvert \Bigr)^2 = \bigl[\E\bigl(\ol{X} - \E\ol{X}\bigr)\bigr]^2,
\] which is
\[
\leq \E \bigl(\ol X - \E \ol X \bigr)^2 = \Var(\ol X) = \frac{\sigma^2}{N}.
\]
Taking the square root gives us the desired result.
\end{mansol}

\begin{mansol}{2.1.4}
Let Z replace $g \sim N(0,1)$. By the law of unconscious statistician, 
\[
\E(Z^2 \I{Z > t}) = \int_t^\infty z^2 \frac{1}{\sqrt{2\pi}} e^{-z^2 / 2}\,dz 
= \frac{1}{\sqrt{2\pi}}\int_t^\infty z \cdot \bigl(z e^{-z^2 / 2}\bigr)\,dz.
\]
Integration by parts gives us \begin{align*}
\int_t^\infty z \cdot \bigl(z e^{-z^2 / 2}\bigr)\,dz & = \bigl[z \cdot \bigl(- e^{-z^2 / 2}\bigr)\bigr]_t^\infty - \int_t^\infty \bigl(- e^{-z^2/2}\bigr)\,dz \\
& = t \bigl(e^{-t^2 / 2}\bigr) + \int_t^\infty \bigl(e^{-z^2/2}\bigr)\,dz.
\end{align*}
Hence \[
\E(Z^2 \I{Z > t}) = t \cdot \frac{1}{\sqrt{2\pi}} e^{-t^2/2} + \Pr(Z > t),
\] and the inequality on the right is obvious by proposition 2.1.2.
\end{mansol}

\begin{mansol}{2.2.7}[Proof of 2.2.6 Hoeffding's inequality for general bounded R.V.]\ \\
We follow what we have done in the special case for Rademacher distribution (2.2.2): for any chosen $\lambda > 0$, 
\begin{align}
\Pr \biggl(\sum_{i=1}^N (X_i - \E X_i) \geq t \biggr) & = \Pr \biggl(\exp\Bigl(\lambda \sum_{i=1}^N (X_i - \E X_i)\Bigr) \geq e^{\lambda t}\biggr) \nonumber \\ 
& \leq \frac{\E \exp\Bigl(\lambda \sum_{i=1}^N (X_i - \E X_i)\Bigr)}{e^{\lambda t}} \nonumber \\
& = e^{-\lambda t} \prod_{i=1}^N \E \exp\bigl(\lambda (X_i - \E X_i)\bigr). \label{eq:1}
\end{align}
We know that $\E (X_i - \E X_i) = 0$. To show the exact bound given in the problem, we have to show that

\begin{centering}
for a zero-mean random variable $X \in [a,b]$, for any $\lambda \in \R$, \begin{equation}
\E \exp(\lambda X) \leq \exp\biggl(\frac{1}{8} \lambda^2 (b - a)^2\biggr). \label{eq:2}
\end{equation}
\end{centering}

This is called the \df{Hoeffding's lemma}. A proof of the inequality with 2 instead of 8 is in example 2.4 of HDS\footnote{\textit{High-Dimensional Statistics} by Wainwright. The proof of the original inequality is outlined in exercise 2.4 of the same book, and is given in full length in \textit{Concentration Inequalities} by Boucheron et al. This proof uses the \df{change of measure} technique.}. The technique used is called the \df{symmetrization argument}, and there is another proof of the inequality with 4 instead of 8 using the same technique at \href{https://marcromani.github.io/2021-05-01-hoeffding-lemma/}{this link}. For the most elementary proof, see \href{https://en.wikipedia.org/wiki/Hoeffding\%27s_lemma}{Wikipedia}, which is given below.

\begin{proof}
Since $e^{\lambda x}$ is convex, we have for $x \in [a,b]$ \[
e^{\lambda x} \leq \frac{b-x}{b-a}e^{\lambda a} + \frac{x-a}{b-a}e^{\lambda b}.
\]
Taking expectation on both sides, and we have \begin{equation}
\E \exp(\lambda X) \leq \frac{b}{b-a} e^{\lambda a} - \frac{a}{b-a} e^{\lambda b} = e^{L(\lambda(b-a))}, \label{eq:3}
\end{equation}
where for all $h \in \R$, $L(h) = - \gamma h + \log(1 - \gamma + \gamma e^h)$, with $\gamma = -\frac{a}{b - a} > 0$ [so that the log is well-defined].

Notice that $L(0) = 0$, and $L'(0) = -\gamma + \frac{\gamma e^h}{1 - \gamma + \gamma e^h}\big|_{h = 0} = 0$. \begin{align*}
L''(h) & = \biggl(\frac{\gamma e^h}{1 - \gamma + \gamma e^h}\biggr)' \\
& = \frac{\gamma e^h}{1 - \gamma + \gamma e^h} - \frac{\gamma^2 e^{2h}}{(1 - \gamma + \gamma e^h)^2} \\
& = t - t^2 \leq 1/4\quad \text{for any }t\in \R, 
\end{align*}
where we let $t = \frac{\gamma e^h}{1 - \gamma + \gamma e^h}$. Now we appeal to Taylor's theorem: for $h \neq 0$, there exists a $\xi$ between $h$ and 0 such that \[L(h) = 0 + 0\cdot h + \frac{L''(\xi)}{2} \cdot h^2 \leq \frac{1}{8} h^2.\] Now let $h = \lambda(b-a)$ and plug it back into \eqref{eq:3}, and we have shown \eqref{eq:2}.
\end{proof}

Now we return to \eqref{eq:1}. For any $t > 0$, \begin{align*}
e^{-\lambda t} \prod_{i=1}^N \E \exp\bigl(\lambda (X_i - \E X_i)\bigr) & \leq e^{-\lambda t} \prod_{i=1}^N \exp\biggl(\frac{N}{8} \lambda^2 \bigl((M_i - \E X_i)-(m_i - \E X_i)\bigr)^2\biggr) \\ 
& = \exp \biggl(\frac{\sum_{i = 1}^N (M_i - m_i)^2}{8}\lambda^2 - t \lambda \biggr)
\end{align*}
The expression in the large parentheses is minimized at $\lambda = \frac{4 t}{\sum(M_i - m_i)^2}$ by $- \frac{2t^2}{\sum(M_i - m_i)^2}$. We can thus conclude that \[
\Pr \biggl(\sum_{i=1}^N (X_i - \E X_i) \geq t \biggr) \leq \exp \biggl(- \frac{2t^2}{\sum_{i=1}^N (M_i - m_i)^2}\biggr),
\]
as desired.
\iffalse
Before this we need to show: for a random variable $Y \in [m,M]$,
\[\Var(Y) \leq \frac 1 4 (M-m)^2.\]

\begin{proof}
We know 
\[\Var(Y) = \E (Y - \E Y)^2 \leq \E \biggl(Y - \frac{M+m}{2}\biggr)^2\]
because $\E Y$ is the minimizer of the MSE. Notice that \begin{align*}
\E \biggl(Y - \frac{M+m}{2}\biggr)^2 & = \frac 1 4 \E [(Y - M) + (Y - m)]^2 \\
& \leq \frac 1 4 \E[(Y - m) - (Y - M)]^2
\end{align*}
because $Y - M \leq 0$. Hence $\Var(Y) \leq \frac 1 4 (M-m)^2.$
\end{proof}

Now we return to proving \eqref{eq:1}. Let $X'$ be an independent copy of $X$. Because here $\E X = \E X' = 0$, \begin{align*}
\E_X \exp(\lambda X) & = \E_X \exp\bigl(\lambda (X - \E_{X'}X')\bigr) \\
& \leq \E_X \E_{X'} \exp\bigl(\lambda (X - X') \bigr),
\end{align*}
by Jensen's inequality. Let $U = X - X'$. $U$ is a zero-mean symmetric\footnote{Clearly $\E U = \E X - \E X' = 0$. Since $X$ and $X'$ are i.i.d., $X$ and $X'$ are exchangeable, i.e., $(X,X') \eqD (X',X)$. Applying the function $g(a,b) = a - b$ to the two random vectors shows that $X - X' \eqD X' - X$, i.e., $U$ is symmetric.} random variable, and is always in $[m - M, M - m]$. Now by Taylor's theorem, for all $u \in [m - M, M - m]$, 
\[
e^{\lambda u} = 1 + (\lambda u) + \frac{\lambda^2 u^2}{2!} + \frac{\lambda^3 u^3}{3!}e^{\lambda \theta}
\]
for some $\theta$ between 0 and $u$. We want to take the expectation of the expression above. By symmetry we know that $\E U^3 = 0 = \E U$. Meanwhile, $\E U^2 = \Var(U) = 2 \Var(X)$. It follows that 
\begin{align*}
\E e^{\lambda U} & = 1 + \frac{\lambda^2}{2} \E U^2\\
& = 1 + \lambda^2
\end{align*}
\fi
\end{mansol}

\begin{mansol}{2.2.8}[Boosting]
Let $X_i = \I{\text{wrong $i$-th classification}}$. Then $\E X_i = \frac{1}{2} - \delta$. Basically we want to bound $\Pr \bigl(\sum_{i=1}^N X_i \geq \frac{N}{2}\bigr)$ above by $\epsilon$. We try to use Hoeffding's inequality 2.2.6:
\begin{align*}
\Pr \biggl(\sum_{i=1}^N X_i \geq \frac{N}{2}\biggr) & = \Pr \biggl(\sum_{i=1}^N X_i - N\Bigl(\frac{1}{2} - \delta \Bigr)\geq N\delta \biggr) \\
& \leq \exp\biggl(-\frac{2 N^2\delta^2}{N}\biggr).
\end{align*}
We may just let $\exp\bigl(-2N \delta^2\bigr) \leq \epsilon$; equivalently this means $N \geq -\frac{1}{2\delta^2} \log(\epsilon)$.
\end{mansol}

\begin{mansol}{2.2.9}
\begin{enumerate}[label=(\alph*)]
    \item Easy; by Chebyshev's inequality, let $N \geq 4 \sigma^2 / \epsilon^2$.
    \item Let $\hat{\mu}_1,\hat{\mu}_2,\ldots,\hat{\mu}_K$ be $K$ weak estimates from part (a). The sample size $N$ is now $\geq 4K \sigma^2/\epsilon^2$.

    Consider $\hat \mu = \operatorname{median}\{\hat \mu_i\}_{i=1}^K$, and let $Y_i = \I{\abs{\mu_i - \mu} > \epsilon}$ for $i$ between 1 and $K$. Then $\hat \mu - \mu > \epsilon$ (resp.\ $< -\epsilon$) implies that more than half of $\mu_i - \mu$ is $> \epsilon$ (resp.\ $< -\epsilon$). To be precise, let $K$ be an odd number, then \[\abs{\hat\mu - \mu} > \epsilon \implies \sum_{i=1}^K Y_i \geq K/2.\] It follows that \[
        \Pr(\abs{\hat \mu - \mu} > \epsilon) \leq \Pr\Bigl(\sum_{i=1}^K Y_i \geq K/2\Bigr) \leq \Pr\Bigl(\sum_{i=1}^K (Y_i - \E Y_i) \geq K/4\Bigr),
    \]
    where the second inequality follows from \[\E Y_i = \Pr(\abs{\mu_i - \mu} > \epsilon) \leq 1/4\] given in part (a). Now by Hoeffding's inequality for bounded random variables (theorem 2.2.6) we have \[
    \Pr\Bigl(\sum_{i=1}^K (Y_i - \E Y_i) \geq K/4\Bigr) \leq \exp(- K / 8).
    \]
    Therefore to let $\Pr(\abs{\hat \mu - \mu} > \epsilon) \leq \delta$, we may just let $\exp(- K / 8) \leq \delta$, or $K \geq 8 \log(\delta^{-1})$. This show that the median estimates from a sample of total size $\geq 32 \log(\delta^{-1}) \sigma^2/\epsilon^2$ gives an $\epsilon$-accurate estimate with probability at least $1-\delta$.
\end{enumerate}
\end{mansol}

\begin{mansol}{2.2.10}
\begin{enumerate}[label=(\alph*)]
\item Easy; by $\int_0^\infty e^{-tx} dx = 1/t$.
\item Easy; just follow the hint.
\end{enumerate}
\end{mansol}

\begin{mansol}{2.3.2}
For $\lambda < 0$, \[\Pr(S_N \leq t) = \Pr\bigl(e^{\lambda S_N} \geq e^{\lambda t}\bigr) \leq e^{-\lambda t} \E \exp(\lambda S_N).\]
Now follow the proof of theorem 2.3.1 and choose $\lambda = \log(t/\mu) < 0$ this time as well. \textit{Note that here $t > 0$.}
\end{mansol}

\begin{mansol}{2.3.3}
Let us see $X \sim \text{Poisson}(\lambda)$ as the limit in distribution of a sum $S_N$ of Bernoulli random variables specified by the Poisson limit theorem. Take limits on the two sides of theorem 2.3.1 and we get \[\Pr(X \geq t) = \lim_{N\to \infty} \Pr(S_N \geq t) \leq \lim_{N\to \infty} e^{-\mu} \Bigl(\frac{e\mu}{t}\Bigr)^t = e^{-\lambda}\Bigl(\frac{e\lambda}{t}\Bigr)^t.\]
\end{mansol}

\begin{mansol}{2.3.5}
    An important log inequality (via Padé approximation) says that 
    \begin{align}
        \frac{2x}{2+x} & \leq \log(1+x) \leq \frac{x}{2}\cdot\frac{2+x}{1+x}\quad \text{for }x\geq 0,\text{ and} \label{eq:4}\\
        \frac{2x}{2+x} & \geq \log(1+x) \geq \frac{x}{2}\cdot\frac{2+x}{1+x}\quad \text{for }-1 < x\leq 0. \label{eq:5}
    \end{align}
    Note the expressions in \eqref{eq:4} and \eqref{eq:5} above all have Taylor expansions $x - \frac{x^2}{2} + O(x^3)$ within their radii of convergence. Also, rewriting $\log(1+x)$ in \eqref{eq:4} by $-\log\bigl(1 - \frac{x}{1+x}\bigr)$ gives us \eqref{eq:5}. Just to verify the inequalities, check the derivatives of the expressions as usual.

    From $\Pr(S_N - \mu \geq \delta \mu) \leq \Bigl(\frac{e^\delta}{(1+\delta)^{1+\delta}}\Bigr)^\mu$, we want to show that $\frac{e^\delta}{(1+\delta)^{1+\delta}} \leq e^{-c_1\delta^2}$ for some $c_1 > 0$. Take the logarithm of the LHS gives \[
    \delta - (1+\delta)\log(1+\delta), \text{ which is } \leq \delta - (1+\delta)\cdot\frac{2\delta}{2+\delta} = -\frac{\delta^2}{2+\delta}
    \]
    Since we let $\delta \leq 1$, we can let $c_1 = 1/3$.

    By exercise 2.3.2 we know $\Pr(S_N - \mu \leq -\delta \mu) \leq \Bigl(\frac{e^{-\delta}}{(1-\delta)^{1-\delta}}\Bigr)^\mu$ given $\delta \in (0,1)$, and when $\delta = 1$ the inequality is trivally true. We want to find some $c_2 > 0$ such that \[
    -\delta - (1-\delta)\log(1-\delta) \leq -c_2 \delta^2.
    \]
    For $-\delta \in [-1,0)$, \[
    \text{LHS} \leq -\delta - (1 -\delta)\cdot \frac{-2\delta}{2-\delta} = -\frac{\delta^2}{2-\delta}.
    \]
    By $-\delta < 0$ we may choose $c_2 = 1/2$. 

    Hence for $\delta \in (0,1]$, \[
    \Pr(\abs{S_N - \mu} \geq \delta \mu) \leq e^{-\frac{1}{2}\mu \delta^2} + e^{-\frac{1}{3}\mu \delta^2} \leq 2 e^{-\frac{1}{2}\mu \delta^2}.
    \]
\end{mansol}

\begin{mansol}{2.4.3}
    We know $d \leq C$. 
\end{mansol}

\begin{mansol}{2.5.1}
    \begin{align*}
        \E \abs{X}^p & = \int_{x = -\infty}^{\infty} \abs{x}^p \frac{1}{\sqrt{2 \pi}}e^{-\frac{1}{2}x^2}\,dx \\
        & = \sqrt{\frac{2}{\pi}}\int_{x=0}^\infty x^p e^{-\frac{1}{2}x^2}\,dx \\
        & = \sqrt{\frac{2}{\pi}}\int_{t=0}^\infty (\sqrt{2t})^p e^{-t}\,d{\sqrt{2t}} \\
        & = \frac{2^{\frac{p+1}{2}}}{\sqrt{\pi}} \int_{t=0}^\infty t^{p/2} e^{-t} \cdot \frac{1}{\sqrt{2}} t^{-\frac{1}{2}}\,dt \\
        & = 2^{p/2}\frac{\int_{t=0}^\infty t^{\frac{p-1}{2}} e^{-t}\,dt}{\sqrt{\pi}} = \bigl(\sqrt{2}\bigr)^p\frac{\Gamma\bigl(\frac{p+1}{2}\bigr)}{\Gamma\bigl(\frac{1}{2}\bigr)}.
    \end{align*}
    Now take the $p$-th root on both sides to get $\nm{X}_{L^p} = \sqrt{2} \biggl[\frac{\Gamma\bigl(\frac{p+1}{2}\bigr)}{\Gamma\bigl(\frac{1}{2}\bigr)}\biggr]^{1/p}$. Note that when $p$ is even, \[\Gamma\Bigl(\frac{p+1}{2}\Bigr) = \frac{p-1}{2}\frac{p-3}{2}\cdots\frac{1}{2}\cdot\Gamma\Bigl(\frac{1}{2}\Bigr),\] and when $p$ is odd, $\Gamma\bigl(\frac{p+1}{2}\bigr) = \bigl(\frac{p-1}{2}\bigr)!$. Both cases show that \[\biggl[\frac{\Gamma\bigl(\frac{p+1}{2}\bigr)}{\Gamma\bigl(\frac{1}{2}\bigr)}\biggr]^{1/p} = O\bigl(((p-1)!!)^{1/p}\bigr) = 
    O\bigl(\bigl(\sqrt{p!}\bigr)^{1/p}\bigr) = O\bigl(\sqrt{p}\bigr),\] and thus $\nm{X}_{L^p} = O(\sqrt{p})$.
\end{mansol}

\begin{mansol}{2.5.4}
    Expand $\E \exp(\lambda X)$ to get \begin{align*}
        \lambda \E X & = \E \exp(\lambda X) - 1 - \sum_{k=2}^\infty \frac{(\E X^k)\lambda^k}{k!} \\
        & \leq \exp(K_5^2 \lambda^2) - 1 - \sum_{k=2}^\infty \frac{(\E X^k)\lambda^k}{k!}
    \end{align*}
    Now consider $\lambda \uparrow 0$ and $\downarrow 0$ respectively. By the uniform convergence and continuity of convergent power series we know that $\lim_{\lambda \to 0}\sum_{k=2}^\infty \frac{(\E X^k)\lambda^k}{k!} = 0$. Hence $\lambda \E X \leq 0$ for both $\lambda \uparrow 0$ and $\downarrow 0$. This shows that $\E X = 0$.
\end{mansol}

\begin{mansol}{2.5.5}
    \begin{enumerate}[label=(\alph*)]
        \item $\E \exp(\lambda^2 X^2) = \frac{1}{\sqrt{2\pi}}\int_{x = -\infty}^\infty \exp\bigl(\bigl(\lambda^2 -\frac 1 2\bigr) x^2\bigr)\,dx$. We have to let $\lambda^2 < 1/2$, i.e., $\lambda \in \bigl(\frac{-1}{\sqrt{2}}, \frac{1}{\sqrt{2}}\bigr)$, so that the improper integral converges and can be evaluated as a Gaussian integral.
        \item $\nm{X}_\infty$ is defined to be $\sup\{M \in \R : \mu(\{\omega : \abs{X(\omega)} > M\}) > 0\}$. Consider \begin{align*}
            \Pr (\abs{X} > M) & = \Pr\bigl(\exp(\lambda^2 X^2) > \exp(\lambda^2 M^2)\bigr) \\
            & \leq \frac{\E \exp(\lambda^2 X^2)}{\exp(\lambda^2 M^2)} \\
            & \leq \exp\bigl(\lambda^2(K^2 - M^2)\bigr).
        \end{align*}
        Choose $M > K$ to make the inequality nontrivial. Since the inequality above holds for all real $\lambda$, take $\lambda \to \infty$ and we see $\Pr(\abs{X} > M) = 0$ for any $M > K$. This shows that $X$ is a bounded random variable.
    \end{enumerate}
\end{mansol}

\begin{mansol}{2.5.7}
    First, \[X = 0 \implies \nm{X}_{\psi_2} = \inf\{t:t>0\} = 0.\] Now assuming $\nm{X}_{\psi_2} = 0$, if we assume that $X$ is not almost surely 0, then $\E X^2 \geq (\E X)^2 > 0$. By Jensen's inequality, given $0<t\leq 1$, \[
    \bigl(e^{\E X^2}\bigr)^{1/t^2} \leq \bigl(\E e^{X^2}\bigr)^{1/t^2} \leq \E \exp(X^2 / t^2).
    \]
    Since $e^{\E X^2} > 1$, we can choose some $t > 0$ such that $\bigl(e^{\E X^2}\bigr)^{1/t^2} > 2$. By the definition of infimum we know that for some $0 < \delta < t$, \[
         \bigl(e^{\E X^2}\bigr)^{1/t^2} \leq \bigl(e^{\E X^2}\bigr)^{1/\delta^2} \leq \E \exp(X^2 / \delta^2) \leq 2,
    \]
    which leads to contradiction. Hence $X = 0$ a.s.

    Homogeneity is clear from definition, and now we prove the triangle inequality. Let $g(x) = \exp(x^2)$, which as a composition of convex functions is convex. Now let $t_X = \nm{X}_{\psi_2} + \epsilon/2$, and $t_Y = \nm{Y}_{\psi_2} + \epsilon/2$ for any $\epsilon > 0$, and we have \[
        g \biggl(\frac{X+Y}{t_X + t_Y}\biggr) \leq \frac{t_X}{t_X + t_Y} g\biggl(\frac{X}{t_X}\biggr) + \frac{t_Y}{t_X + t_Y} g\biggl(\frac{Y}{t_Y}\biggr)
    \]
    Now take expectation on both sides [we can certainly ignore the case $t_X$ or $t_Y = 0$]. It follows that \[\E g \biggl(\frac{X+Y}{t_X + t_Y}\biggr) \leq 2.\]
    This tells us that $\nm{X}_{\psi_2} + \nm{Y}_{\psi_2} + \epsilon = t_X + t_Y \in \{t>0 : \E g(\frac{X+Y}{t}) \leq 2\}$. Therefore \[
        \nm{X+Y}_{\psi_2} \text{ exists and is } \leq \nm{X}_{\psi_2} + \nm{Y}_{\psi_2} + \epsilon.
    \]
    Since $\epsilon$ is arbitrary, the triangle inequality follows.
\end{mansol}

\begin{mansol}{}
    
\end{mansol}
\end{document}