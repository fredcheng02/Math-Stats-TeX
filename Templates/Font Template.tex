% Use LuaTeX or XeTeX
% \usepackage{unicode-math}
% \setmainfont{XCharter}
% \setmathfont{XCharter-Math.otf}[CharacterVariant={6}] % phi replaced by varphi
\usepackage[CharacterVariant={6}]{xcharter-otf}

\usepackage{unicode-math}
\setmainfont{EBGaramond-Regular}[
  BoldFont = EBGaramond-Bold,
  ItalicFont = EBGaramond-Italic,
  BoldItalicFont = EBGaramond-BoldItalic
]
\setmathfont{Garamond Math}[Scale=MatchUppercase]

\usepackage[CharacterVariant={6}]{euler-math}
\setmainfont{CMU Concrete}

\usepackage{concmath-otf}

% only in pdfTeX
% \usepackage{sansmathfonts}
% \usepackage[T1]{fontenc}
% \renewcommand*\familydefault{\sfdefault} %% Only if the base font of the document is to be sans serif

\usepackage{libertine}
\usepackage{libertinust1math}
\usepackage[T1]{fontenc}

\usepackage[sb]{libertine} % or \usepackage[sb]{libertinus}
\usepackage[T1]{fontenc}
\usepackage{textcomp}
\usepackage[varqu,varl]{zi4}% inconsolata for mono, not LibertineMono
\usepackage[amsthm]{libertinust1math} % slanted integrals, by default
% \usepackage[scr=boondoxo,bb=boondox]{mathalpha} %Omit bb=boondox for default libertinus bb

% \usepackage{newtxtext,newtxmath}



% \usepackage[T1]{fontenc}
% \usepackage{newpxtext,eulerpx}

% \usepackage[scaled=0.95]{inconsolata}
% \renewcommand{\ttdefault}{inconsolata}

% \usepackage{bbm}; use \mathbbm for blackboard bold for Computer Modern

\usepackage{newpxtext,eulerpx}
% \usepackage{mathpazo}
% \usepackage{newpxtext,newpxmath}
\usepackage[scaled=1.008]{inconsolata}
\usepackage[scaled=0.842]{berasans}
\renewcommand{\mathbf}{\boldsymbol}