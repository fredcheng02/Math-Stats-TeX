% Use LuaTeX or XeTeX
% \usepackage{unicode-math}
% \setmainfont{XCharter}
% \setmathfont{XCharter-Math.otf}[CharacterVariant={6}] % phi replaced by varphi
% \usepackage[CharacterVariant={6}]{xcharter-otf}

% \usepackage{unicode-math}
% \setmainfont{TeX Gyre Termes}
% \setmathfont{TeX Gyre Termes Math}

\usepackage{unicode-math}
\setmainfont{EBGaramond-Regular}[
  BoldFont = EBGaramond-Bold,
  ItalicFont = EBGaramond-Italic,
  BoldItalicFont = EBGaramond-BoldItalic
]
\setmathfont{Garamond Math}[Scale=MatchUppercase]

\usepackage{ccfonts}
\usepackage[euler-digits]{eulervm}
\renewcommand{\mathbf}{\mathbold}

% if you want bold Concrete: 
% \usepackage[math-style=upright]{unicode-math}
% \setmainfont{CMU Concrete}
% \setmathfont{Euler Math}

% \usepackage{concmath-otf}

% \usepackage{fourier} for pdfTeX
% \usepackage{fourier-otf} for XeTeX and LuaTeX

% only in pdfTeX
% \usepackage{sansmathfonts}
% \usepackage[T1]{fontenc}
% \renewcommand*\familydefault{\sfdefault} %% Only if the base font of the document is to be sans serif

% use the following locally when mtpro2 is installed
% \usepackage{newtxtext}
% \usepackage[lite]{mtpro2}
% Alter \fontdimen20 in the math symbol font
% \DeclareFontShape{LMP2}{mtt}{m}{n}{%
%   <-7> mt2syf
%   <7-9> mt2sys
%   <9-> mt2syt
% }{% Default is 2.39. We change it to 2.38
%   \fontdimen20\font=2.38\fontdimen6\font
% }

% \usepackage{mathspec}
% \setmainfont{Liberation Sans}
% \setmathfont(Digits,Latin,Greek){Liberation Sans}

\usepackage{libertinus}
\usepackage{libertinust1math}
\usepackage[T1]{fontenc}

\usepackage[sb]{libertine} % or \usepackage[sb]{libertinus}
\usepackage[T1]{fontenc}
\usepackage{textcomp}
\usepackage[varqu,varl]{zi4}% inconsolata for mono, not LibertineMono
\usepackage[amsthm]{libertinust1math} % slanted integrals, by default
% \usepackage[scr=boondoxo,bb=boondox]{mathalpha} %Omit bb=boondox for default libertinus bb

% \usepackage{newtxtext,newtxmath}

% \usepackage{newpxtext,newpxmath}

% \usepackage[scaled=0.95]{inconsolata}
% \renewcommand{\ttdefault}{inconsolata}

% \usepackage{dsfont}; use \mathbbm for blackboard bold for Computer Modern

\usepackage{mathpazo}
\usepackage{eucal}
\usepackage[scaled=1.008]{inconsolata}
\usepackage[scaled=0.842]{berasans}

% when using unicode-math, operatorname should be defined using `AtBeginDocument`, after all fonts are known