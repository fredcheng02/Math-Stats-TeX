\documentclass{article}
\usepackage[utf8]{inputenc}
\usepackage[T1]{fontenc}
\usepackage{amsmath,amssymb,tikz-cd,mathtools}
\usepackage{enumitem}
\usepackage{fullpage,setspace}
\usepackage[pdfusetitle,bookmarksnumbered]{hyperref}
\newcommand{\df}[1]{\textbf{\textsf{#1}}}
\newcommand{\R}{\mathbb{R}}
\newcommand{\C}{\mathbb{C}}
\newcommand{\Z}{\mathbb{Z}}
\newcommand{\F}{\mathbb{F}}
\renewcommand{\Re}{\operatorname{Re}}
\renewcommand{\Im}{\operatorname{Im}}
\newcommand{\s}{\operatorname{span}}
\newcommand{\n}{\operatorname{null}}
\renewcommand{\r}{\operatorname{range}}
\renewcommand{\d}{\dim}
\newcommand{\conj}[1]{\overline{#1}}
\newcommand{\inp}[2]{\langle #1, #2 \rangle}
\newcommand{\nm}[1]{\| #1 \|}
\newcommand{\LV}{\mathcal{L}(V)}
\newcommand{\LVW}{\mathcal{L}(V,W)}
\newcommand{\M}{\mathcal{M}}
\newcommand{\PF}{\mathcal{P}(\F)}
\newcommand{\bv}{v_1,\dots,v_n}
\newcommand{\bw}{w_1,\dots,w_n}
\renewcommand{\phi}{\varphi}

\title{MATH 110 Supplement on Matrix Theory}
\author{Cheng, Feng}

\onehalfspacing
\setlength{\parskip}{0em}
\setlist{listparindent=\parindent,parsep=0pt}

\begin{document}

\makeatletter
\begin{center}
    {\Large \@title}
    \vspace{1em}
    \\ \@author
    \vspace{0.5em}
\end{center}
\makeatother
\begin{itemize}
    \item You should know that a real matrix $A$ with $A^{-1} = A^t$ is called an \df{orthogonal matrix}, and a complex matrix $A$ with $A^{-1} = A^*$ is called a \df{unitary matrix}, by the equivalence between square matrices and operators, and between $S^{-1} = S^*$ and $S$ being an isometry.
\end{itemize}


\end{document}
