\documentclass[11pt]{article}
\usepackage[utf8]{inputenc}
\usepackage[T1]{fontenc}
\usepackage{amsmath,amssymb,amsthm}
\usepackage{tikz-cd,mathtools}
\usepackage{enumitem}
\usepackage{fullpage,setspace,microtype,soul,framed}
\usepackage[pdfusetitle,bookmarksnumbered]{hyperref}
\newcommand{\lk}[2]{\hyperlink{subsection.#1.#2}{\S#1.#2}}
\newcommand{\rmk}{\noindent\textit{Remark. }}
\newcommand{\df}[1]{\ul{\textit{\textsf{#1}}}} % definition
\newcommand{\la}{\langle}
\newcommand{\ra}{\rangle}
% \usepackage{bbm} use \mathbbm for blackboard bold
\newcommand{\N}{\mathbf{N}}
\newcommand{\Z}{\mathbf{Z}}
\newcommand{\Q}{\mathbf{Q}}
\newcommand{\R}{\mathbf{R}}
\newcommand{\C}{\mathbf{C}}
% \newcommand{\F}{\mathbf{F}} bold field
\renewcommand{\Re}{\operatorname{Re}}
\renewcommand{\Im}{\operatorname{Im}}
\newcommand{\where}{\,|\,}
\newcommand{\conj}[1]{\overline{#1}}
\newcommand{\clos}[1]{\overline{#1}}
\newcommand{\inp}[2]{\langle #1, #2 \rangle}
\newcommand{\nm}[1]{\|#1\|}
\newcommand{\abs}[1]{\lvert #1 \rvert}
\newcommand{\sgn}{\operatorname{sgn}}
\newcommand{\diff}{\,d}% or {\mathop{}\!\mathrm{d}} as an operator
\newcommand{\dx}{\diff x}
\renewcommand{\phi}{\varphi}

\title{MATH 185 Companion}
\author{Cheng, Feng}

\onehalfspacing
\setlength{\parskip}{0em}
\setlist{listparindent=\parindent,parsep=0pt}

\begin{document}

\makeatletter
\begin{center}
    {\Large \@title}
    \\ to Priestley's \textit{Introduction to Complex Analysis}, second edition
    \vspace{0.5em}
    \\ \@author
    \vspace{-0.5em}
\end{center}
\makeatother

\section{The complex plane}
\begin{itemize}
    \item Products, quotients, and powers look the best in the polar form. Use the polar form in calculations.
    \item $\abs{z}^2 = z\conj{z}$ is an important formula. Also, $\Re z\conj{w} = \Re \conj{z} w$, so that $\abs{z+w}^2 = \abs{z}^2 + \abs{w}^2 + 2 \Re z\conj{w}$. (This is used to prove the triangular inequality for complex numbers.)
\end{itemize}

\section{Geometry in the complex plane}
\begin{itemize}
    \item There are multiple ways of expressing an arbitrary line in $\C$:
    \begin{enumerate}[label=\alph*)]
        \item $\abs{z - \alpha} = \abs{z - \beta}$ for $\alpha \neq \beta$.
        
        This is geometrically obvious.
        \item $\Re az = c$ for $a \neq 0$ and $c \in \R$. (You may replace $\Re$ by $\Im$.)
        
        Let $a = u + iv \neq 0$ and $z = x + iy$. Then $ux - vy = c \in \R$ with $u$ and $v$ not simultaneously 0. This is exactly the expression for a line in $\C$.
        
        This fact may be also understood geometrically in terms of polar coordinates. The line represented (by the set of $z$'s) revolves counterclockwise by $\arg a$ about the point $z_1$ on the line with $\Re z_1$ and becomes the vertical line at $c$.
        \item $\conj{B}z + B \conj{z} + C = 0$ for $B \neq 0$.
        
        This is the same as $ \Re B\conj{z} = -C/2$.
    \end{enumerate}
    \item The set $\mathcal{M}$ of Möbius transformations from $\widetilde{\C}$ to $\widetilde{\C}$ forms a group under function composition. This group is isomorphic to the projective general linear group (PGL) and the projective special linear group (PSL) under matrix multiplication.
\end{itemize}

\section{Topology and analysis in the complex plane}
\begin{itemize}
    \item The definition of connectedness on page 37 is wrong. A subspace $S$ of $\C$ is \df{disconnected} if it can be expressed as the disjoint union of two  sets \textit{open in $S$}.
\end{itemize}

\end{document}