\documentclass[11pt]{article}
\usepackage{fontspec}
\usepackage[margin=1in]{geometry}
\usepackage{amsmath,amsthm}
% \usepackage{amssymb,mathtools}
\usepackage{stmaryrd}
\usepackage{enumitem}
\usepackage{setspace,soul,framed}
\usepackage[pdfusetitle,bookmarksnumbered]{hyperref}

\setmainfont{Tex Gyre Pagella}
\usepackage{euler-math}

\newcommand{\lk}[2]{\hyperlink{subsection.#1.#2}{\S#1.#2}}
\newcommand{\rmk}{\noindent\textit{Remark. }}
\newcommand{\df}[1]{\textit{\textsf{#1}}} % definition
\newcommand{\la}{\langle}
\newcommand{\ra}{\rangle}
\newcommand{\lb}{\llbracket}
\newcommand{\rb}{\rrbracket}
\newcommand{\N}{\mathbf{N}}
\newcommand{\Z}{\mathbf{Z}}
\newcommand{\Q}{\mathbf{Q}}
\newcommand{\R}{\mathbf{R}}
\newcommand{\C}{\mathbf{C}}
\newcommand{\eC}{\widetilde{\C}}
\newcommand{\where}{\,|\,}
\newcommand{\conj}[1]{\overline{#1}}
\newcommand{\clos}[1]{\overline{#1}}
\newcommand{\inp}[2]{\langle #1, #2 \rangle}
\newcommand{\nm}[1]{\lVert #1 \rVert}
\newcommand{\abs}[1]{\lvert #1 \rvert} 

\AtBeginDocument{%
    \renewcommand{\Re}{\operatorname{Re}}
    \renewcommand{\Im}{\operatorname{Im}}
    \newcommand{\Bd}{\operatorname{Bd}}
}

\makeatletter
\let\@subtitle\@empty % default value
\protected\def\subtitle#1{\gdef\@subtitle{\\ #1}}
\renewcommand{\maketitle}{
    \begin{center}
        {\Large \@title}
        \@subtitle
        \vspace{0.5em}
        \\ \@author
        \vspace{-0.5em}
    \end{center}
}
\makeatother

\title{MATH 185 Companion}
\author{Cheng, Feng}
\subtitle{to Priestley's \textit{Introduction to Complex Analysis}, second edition}

\onehalfspacing
\setlength{\parskip}{0em}
\setlist{listparindent=\parindent,parsep=0pt}


\begin{document}
\maketitle

\section{The complex plane}
\begin{itemize}
    \item Products, quotients, and powers look the best in the polar form. Use the polar form in calculations.
    \item $\abs{z}^2 = z\conj{z}$ is an important formula. Also, $\Re z\conj{w} = \Re \conj{z} w$, so that $\abs{z+w}^2 = \abs{z}^2 + \abs{w}^2 + 2 \Re z\conj{w}$. (This is used to prove the triangular inequality for complex numbers.)
\end{itemize}

\section{Geometry in the complex plane}
\begin{itemize}
    \item There are multiple ways of expressing an arbitrary line in $\C$:
    \begin{enumerate}[label=\alph*)]
        \item $\abs{z - \alpha} = \abs{z - \beta}$ for $\alpha \neq \beta$.
        
        This is geometrically obvious.
        \item $\Re az = c$ for $a \neq 0$ and $c \in \R$. (You may replace $\Re$ by $\Im$.)
        
        Let $a = u + iv \neq 0$ and $z = x + iy$. Then $ux - vy = c \in \R$ with $u$ and $v$ not simultaneously 0. This is exactly the expression for a line in $\C$.
        
        This fact may be also understood geometrically in terms of polar coordinates. The line represented (by the set of $z$'s) revolves counterclockwise by $\arg a$ about the point $z_1$ on the line with $\Re z_1$ and becomes the vertical line at $c$.
        \item $\conj{B}z + B \conj{z} + C = 0$ for $B \neq 0$.
        
        This is the same as $ \Re B\conj{z} = -C/2$.
    \end{enumerate}
    \item The set $\mathcal{M}$ of Möbius transformations from $\eC$ to $\widetilde{\C}$ forms a group under function composition. This group is isomorphic to the projective general linear group (PGL) and the projective special linear group (PSL) under matrix multiplication.
\end{itemize}

\section{Topology and analysis in the complex plane}
\begin{itemize}
    \item For topological reasons we define the $D(\infty; r)$ on $\eC$ as $\{z \in \C : \abs{z} > r\} \cup \{\infty\}$, and a set $S$ on $\eC$ as open if there is a disc around every point of $S$ contained in $S$. It is clear that $\C$ and all open subsets of $\C$ are open in the new space $\eC$ still.
    \item The definition of connectedness on page 37 is wrong. A subspace $S$ of $\C$ is \df{disconnected} if it can be expressed as the disjoint union of two  sets \textit{open in $S$}.
    % \item Let's prove that the union of two open balls is convex if and only if one is contained in the other.
    
    % Consider the two balls $B(a;r_1)$ and $B(b;r_2)$. For $y$ in the former and $z$ in the latter, if $[y,z] \subseteq B(a;r_1) \cup B(b;r_2)$, then equivalently every $x = ty + (1-t)z$, where $t \in [0,1]$, should satisfy \[\nm{x-a} < r_1 \quad \text{or} \quad \nm{x-b} < r_2.\] Looking at the first equation, we see that $\nm{ty + (1-t)z - a} = \nm{t(y-a) + (1-t)(z-a)} < r_1$
\end{itemize}

\section{Paths}
\begin{itemize}
    \item The book defines the boundary of $S$ [$\Bd S$] as $\clos{S} \cap \clos{S^c}$. The \df{boundary of $S$} is better defined as $(\operatorname{Int} S \cup \operatorname{Ext} S)^c$ or $\clos{S}-\operatorname{Int} S$. The definitions are clearly equivalent because the complement of $\operatorname{Int} S$ [resp.\ $\operatorname{Ext} S$], the largest open set contained in $S$ [resp.\ $S^c$], must be the smallest closed set containing $S^c$ [resp.\ $S$], i.e, $\clos{S^c}$ [resp.\ $\clos{S}$].
\end{itemize}

\section{Holomorphic functions}

\section{Complex series and power series}

\section{A cornucopia of holomorphic functions}

\setcounter{section}{9}
\section{Integration in the complex plane}

\section{Cauchy's theorem: basic track}
\begin{itemize}
    \item $\int_{[z,z+h]}1\,dw=\int_{z}^{z+h}1\,dw=h$ 
\end{itemize}

\end{document}